
\beforepreface
\prefacesection{Abstract}
\label{sec:abstract}

This thesis covers the design and implementation of Legion,
a new programming model and runtime system for targeting
distributed heterogeneous machine architectures. Legion
introduces {\em logical regions} as a new abstraction for 
describing the structure and
usage of program data. We describe how logical regions 
provide a mechanism for applications to express important 
properties of program data, such as locality and independence, 
that are often ignored by current programming systems. We also
show how logical regions allow programmers to scope the usage of 
program data by different computations. The explicit nature of 
logical regions makes these properties of programs manifest, 
allowing many of the challenging burdens of parallel programming,
including dependence analysis and data movement, to be off-loaded 
from the programmer to the programming system.

Logical regions also improve the programmability and portability 
of applications by decoupling the specification of a program 
from how it is mapped onto a target architecture. Logical
regions abstractly describe sets of program data without
requiring any specification regarding the placement or layout of data.
To control decisions about the placement of computations
and data, we introduce a novel {\em mapping interface}
that gives an application programmatic control over mapping 
decisions at runtime. Different implementations
of the mapper interface can be used to port applications
to new architectures and to explore alternative mapping
choices. Legion guarantees that the decisions made through 
the mapping interface are independent of the correctness 
of the program, thus facilitating easy porting and tuning 
of applications to new architectures with different performance
characteristics.

Using the information provided by logical regions,
an implementation of Legion can automatically
extract parallelism, manage data movement, and infer
synchronization. We describe the algorithms and data
structures necessary for efficiently performing these
operations. We further show how the Legion runtime can 
be generalized to operate as a distributed system, making 
it possible for Legion applications to scale well.
As both applications and machines continue to become more 
complex, the ability of Legion to relieve application 
developers of many of the tedious responsibilities
they currently face will become increasingly important. 

To demonstrate the performance of Legion, we port a
production combustion simulation, called S3D, to Legion. 
We describe how S3D is implemented within the Legion
programming model as well as the different mapping
strategies that are employed to tune S3D for runs
on different architectures. Our performance results
show that a version of S3D running on Legion is 
nearly three times as fast as comparable state-of-the-art
versions of S3D when run at 8192 nodes on the number
two supercomputer in the world.

