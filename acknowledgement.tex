
\prefacesection{Acknowledgement}
\label{sec:acknowledgement}

% Alex
% Bill, Pat, Kunle, Juan, Eric
% Pat McCormick
% Sam Gutierez, Charles, Chao, Shima
% Jackie, Hemanth, Ramanan, Tianfeng, Ankit
% James, Curt, Ted
% Adam, Peter, Tom, Isil, Rahul, Manolois, John, Wonchan, Zhihao
% Eric, Zach
% Katherine, Kshipra
% Sean and Elliott
% Family

This thesis, and more importantly, the Legion runtime, 
would not have been possible without the help and support
of many people. Foremost on this list is my advisor, Alex
Aiken. Alex sought me out my first year at Stanford and 
encouraged me to pursue work on programming systems 
despite knowing that my primary training was in computer
architecture. Through many setbacks and obstacles along
the way, Alex remained patient as I developed the skills 
necessary to become a true systems programmer. Alex 
fostered an environment that emphasized solving important
problems regardless of the risks, instead of working 
on incremental improvements. Time and again, this atmosphere has 
proven crucial for enabling projects like Legion to flourish, 
and I feel privileged to have been a part of it.

% Other professors
I would also like to thank the other professors who
have made my time at Stanford so successful. Bill Dally gave 
me my start at Stanford working on the Sequoia project, inspiring 
many of the ideas in Legion. Pat Hanrahan articulated
the domain specific language story which has driven much of
the design of Legion. By running the Pervasive Parallelism
Lab, Kunle Olukotun facilitated many of the personal connections
necessary for advertising Legion. Both Juan Alonso and Eric Darve
have been early adopters and contributed significantly to the
development of Legion.

A significant portion of this thesis is a result of direct
collaboration with scientists from the national labs as well 
as departments beyond computer science here at Stanford.
Pat McCormick from Los Alamos National Lab was the earliest
champion of Legion and recognized its potential from the start. 
To this day he has remained our most vocal 
supporter and has guided us through the sometimes turbulent
waters of the supercomputing community. Also from Los Alamos,
Sam Gutierrez, Charles Ferenbaugh, and Kei Davis were some of 
the earliest Legion users and provided valuable feedback on the 
initial designs. Hemanth Kolla and Ankit Bhagatwala from Sandia 
National Lab were very generous with their time in explaining 
the science and implementation of S3D. Ramanan Sankaran from Oak Ridge
National Lab provided both code and advice for running
experiments on Titan. Shima Alizadeh and Chao Chen from the
Stanford mechanical engineering department have both been
very supportive early users of Legion. Special thanks go to 
Jackie Chen, from Sandia National Lab, who trusted us to use her production S3D 
code as the primary vehicle for showcasing Legion when no others 
were willing to take the risk. 

% Grad students
Surviving graduate school would not have been possible 
without the support of my fellow graduate students. Conversations,
both technical and recreational, over many years with Tom Dillig, 
Isil Dillig, Rahul Sharma, John Clark, Manolis Papadakis, 
Wonchan Lee, and Zhihao Jia have made me a better computer scientist
and a better person. Adam Oliner and Peter Hawkins were outstanding 
mentors throughout my early years of graduate school. James Balfour, 
Curt Harting, and Ted Jiang made for the best company and kept me grounded 
in my hardware roots. Kshipra Bhawalkar has been my constant
voice of wisdom for ten years through both undergraduate and graduate school.
Katherine Breeden voluntarily proofread this entire thesis and instigated 
many early-morning runs which often proved vital to the maintenance
of my sanity. Elliott Slaughter was the first person brave
enough to join the Legion project and patiently suffered through
many of my bugs while giving gentle feedback, for which I will
always be grateful. My academic brothers, Eric Schkufza and Zach DeVito,
both journeyed with me through the many trials of graduate school
and there is no way I would have made it through without them.

% Sean
From the beginning, the creation and development of Legion has been
a joint partnership with Sean Treichler. I know for certain that 
Legion would not exist in the form that it does today without our 
combined effort. Working with Sean has resulted in one of the most
productive and creative periods of my life, due in a large part to the 
vast knowledge and experience he was willing to share with me. I'm
confident saying that I will never engage in another collaboration 
like the one we have had.

% Family
Since this is the final stage of my formal education, I want to thank
my family for their enduring love and support throughout my 
journey. My parents Steven and Kristine Bauer have been there 
for me from my first day of school, twenty three years ago, to today. 
It has been a long and arduous endeavor at times, but their light 
has always guided me along the path. I would also like to especially
thank my aunt and uncle, 
Tom and Colleen Kistler, for all their support over the years; it
has truly been like having a second set of parents. Lastly, my
brother Rick Bauer has always been there for me, and has been
instrumental in reminding me that there exists a natural world 
beyond the thought spaces in which I so frequently find myself lost.

To all my family and friends: thank you for everything! Thank you.

\afterpreface
